``Galois extension \(F\) of  \(K\) is a field for which the fixed field of the Galois group \(Aut_K^F\) is \(K\) itself'' \cite{hunger}.\\
But for what extension field \(F\) of \(K\) the Galois group keeps the ``base field \(K\)'' fixed? What is the structure of Galois field extension and how do we construct(obtain) it?


\section{Splitting Field}
Since, for \(F=K(u)\), any \(\sigma \in Aut_K^F\) is ``completely determined'' \cite{hunger} by its action on \(u\). Any algebraic Galois extension of \(K\) is generated by all roots \(u\) of a polynomial \(f \in K[x]\) \cite{hunger}.

\begin{definition} \cite{hunger}
  Such a minimal field \(F\) where a polynomial \(f \in K[x]\) ``splits into linear factors'' and thus ``contains all roots of \(f(x)\)'' is called a splitting field of \(f\) over \(K\).
\end{definition}
Thus, an algebraic Galois extension is going to be characterized by ``a splitting field of a polynomial over the base field'' \cite{hunger}.
\vspace{3mm}

\begin{theorem} \cite{hunger} [Existence of a Splitting field]
  If \(K\) is a field and \(f \in K[x]\) has degree \(n \geq 1\), then there exists a splitting field \(F\) of \(f\) with \([F:K] \leq n!\).
  \end{theorem}

\subsection{Algebraic Closure of a Field}
``A field \(F\) is said to be algebraically closed if every non-constant polynomial \(f \in F[x]\) has a root in \(F\)'' \cite{hunger}.
For example the field of complex number \(\mathbb{C}\) is algebraically closed.\\[3mm]

\begin{definition} \cite{hunger}
An extension field \(F\) of a field \(K\) is said to be algebraic closure of \(K\) if,
\begin{enumerate}
\item[i)] F is algebraically closed and,
  \item[ii)] F is algebraic over \(K\).
  \end{enumerate}
\end{definition}


So, \(\mathbb{C}\) is algebraically closed field but is not an algebraic closure of \(\mathbb{Q}\) because \(\mathbb{C}\) is not algebraic over \(\mathbb{Q}\) \cite{hunger}.
But \(\mathbb{C}\) is an algebraic closure of \(\mathbb{R}\).\\
This shows algebraic closure is an special case of a splitting field.

\section{Separable Extension}
``An irreducible polynomial \(f \in K[x]\) is said to be separable if in some splitting field of \(f\) over \(K\) every root of \(f\) is a simple root'' \cite{hunger}.\\
An algebraic element \(u \in F\) is said to be ``separable over \(K\) provided its irreducible polynomial is separable''\cite{hunger}.
\begin{definition} \cite{hunger}
  If every element of \(F\) is separable over \(K\), then \(F\) is said to be a separable extension of \(K\).
\end{definition}
\vspace{3mm}
\noindent
\textbf{Characteristic of a Separable extension}
\begin{remark} \cite{hunger}
  Every algebraic extension field of a field of characteristic \(0\) is separable.
  \end{remark}
If a polynomial \(f \in K[x]\) is separable over \(K\) then it has no multiple roots in any splitting field of \(f\) over \(K\) \cite{hunger}. ``This shows that an irreducible polynomial in \(K[x]\) is separable if and only if its derivative is nonzero'' \cite{hunger}. This holds true for every field of characteristic zero \cite{hunger}.

\section{Galois extension}
\begin{theorem} \cite{hunger}
  \(F\) is algebraic and Galois over \(K\) if and only if \(F\) is separable over \(K\) and \(F\) is a splitting field over \(K\) of a set \(S\) of polynomials in \(K[x]\).
  \end{theorem}

This proves the \textbf{Generalized Fundamental Theorem of Galois Theory},\\
which sates the \textit{Fundamental Theorem of Galois Theory} still holds if the extension field \(F\) is not finite dimensional as-well i.e, if ``\(F\) is algebraic and Galois over \(K\)'' \cite{hunger}.
