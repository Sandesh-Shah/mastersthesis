\begin{definition}
  The Galois group of a polynomial \(f \in K[x]\) is the group \(Aut_K^F\), where \(F\) is a splitting field of \(f\) over \(K\).
  \end{definition}

  \begin{theorem}
  Let \(G\) be a Galois group of a polynomial \(f \in K[x]\).
\begin{enumerate}
\item[i)] \(G\) is isomorphic to a subgroup of some symmetric group \(S_n\)..
  \item[ii)] If \(f\) is separable of degree \(n\), the \(n\) divides \(|G|\) and \(G\) isomorphic to a transitive subgroup of \(S_n\).
  \end{enumerate}
\end{theorem}
Because the Galois group \(Aut_K^F\) is a group of automorphisms of F which is given by the permutations of the roots.
So, the Galois group of a polynomial is identified with the subgroup of \(S_n\).\\

\begin{corollary}
\begin{enumerate}
\item[i)] If the degree of \(f\) is \(2\) then its Galois group \(G \cong {\mathbb{Z}}_2\).
  \item[ii)] If the degree of \(f\) is \(3\) then its Galois group \(G\) is either \(S_3\) or \(A_3\).
  \end{enumerate}
\end{corollary}


\section{Galois Group of Cubic polynomials}
\begin{definition}
  Let \(K\) be a field with \(char K \neq 2\) and \(f \in K[x]\) a polynomial of degree \(n\) with \(n\) distinct roots \(u_1,u_2,...,u_n\) in some splitting field \(F\) of \(f\) over \(K\). Let \(\Delta = \prod\limits_{i<j}(u_i-u_j) = (u_1-u_2)(u_1-u_3)...(u_{n-1}-u_n) \in F\).\\
  The discriminant of \(f\) is the element \(D= {\Delta}^2\).
\end{definition}

\begin{theorem}
\begin{enumerate}
\item[i)] The discriminant \({\Delta}^2\) of \(f\) actually lies in \(K\).
  \item[ii)] For each \(\sigma \in Aut_K^F < S_n, \sigma\) is an even[resp. odd] if and only if \(\sigma(\Delta) = \Delta\)[resp. \(\sigma(\Delta) = - \Delta\)].
  \end{enumerate}
\end{theorem}
Since \({\Delta}^2 \in K\) and \(\Delta \in F\), and \(K(Delta)\) is a stable intermediate; in the Galois correspondence the subfield \(K(\Delta)\) corresponds to the subgroup \(G \cap A_n\). In particular, \(G\) consists of even permutations if and only if \(\Delta \in K\).

\begin{corollary}
  If \(f\) is a separable polynomial of degree \(3\), then the Galois group of \(f\) is \(A_3\) if and only if the discriminant of \(f\) is the square of an element of \(K\).
\end{corollary}

\begin{theorem}
  Let \(K\) be a field of \(char K \neq 2,3 \). If \(f(x)=x^3+bx^2+cx+d \in K[x]\) has three distinct roots in some splitting field, then the polynomial \(g(x)=f(x-b/3) \in K[x]\) has the form \(x^3+p^2+q\) and the discriminant of \(f\) is \(-4p^3-27q^2\) ~\cite{hunger}.
\end{theorem}

\section{Galois Group of Quartic polynomials}
\begin{definition}[Resolvant Cubic of a Quartic]
Let \(K, f, F, u_i, V,\) and \(G=Aut_K^F<S_4\) be as in the preceding paragraph and \(\alpha=u_1u_2+u_3u_4,\) \(\beta=u_1u_3+u_2u_4,\) \(\gamma=u_1u_4+u_2u_3\).
The polynomial \( (x- \alpha)(x- \beta)(x- \gamma) \) is called the resolvant cubic of \(f\). The resolvant cubic is actually a polynomial over \(K\).\\
\end{definition}
Now under the Galois correspondence the subfield \(K(\alpha, \beta, \gamma)\) corresponds to the normal subgroup \(V \cap G\) because \(K(\alpha,\beta,\gamma)\) is a splitting field of the resolvant cubic
whose Galois group is a subgroup of \(S_3\) and only normal subgroup of \(N\) of \(S_4\) with \(|N| \leq 6\) is \(V\).\\
Hence \(K(\alpha, \beta, \gamma)\) is Galois over \(K\) and \(Aut_K^{K(\alpha, \beta, \gamma)} = G/(G \cap V\).

\begin{remark} If \(K\) is a field and \(f = x^4+bx^3+cx^2+dx+e \in K[x]\), then the resolvant cubic of \(f\) is the polynomial \(x^3-cx^2+(bd-4e)x-b^2e+4ce-d^2 \in K[x]\).\\ \\
\end{remark}

\begin{theorem}
  Let \(K\) be a field and \(f \in K[x]\) a separable quartic with Galois Group \(G\). Let \(\alpha, \beta, \gamma\) be the roots of the resolvant cubic of \(f\) and let \(m= [K(\alpha, \beta, \gamma) : K]\) then,
\begin{enumerate}
\item[i)] \(m=6 \Longleftrightarrow G=S_4\);
\item[ii)] \(m=3 \Longleftrightarrow G=A_4\);
\item[iii)] \(m=1 \Longleftrightarrow G=V\);
\item[iv)] \(m=2 \Longleftrightarrow G=D_4\) or \(G={\mathbb{Z}}_4\); in this case \(G={\mathbb{Z}}_4\) if \(f\) is irreducible over \(K(\alpha, \beta, \gamma)\) and \(G={\mathbb{Z}}_4\) otherwise.
  \end{enumerate}
\end{theorem}
This is because we have \([K(\alpha,\beta,\gamma):K] = |G/G \cap V|\).

\section{Galois Group of some Polynomials}

\begin{example}
The polynomial is \(f(x)=x^4-2 \in \mathbb{Q}[x]\).\\
This polynomial is irreducible and so separable over \(\mathbb{Q}\). The resolvant cubic is \(x^3+8x = x(x+2i\sqrt{2})(x-2i\sqrt{2})\) and 
\(\mathbb{Q}(\alpha,\beta, \gamma)=\mathbb{Q}(i\sqrt{2})\) has dimension \(2\) over \(mathbb{Q}\).\\
\(x^4-2\) is irreducible over \(\mathbb{Q}(i\sqrt{2})\) because \(\sqrt[\leftroot{-3}\uproot{3}4]{2} \not\in \mathbb{Q}(i\sqrt{2})\).
Therefore the Galois group \(G \cong D_8\).
\end{example}








%%% Local Variables:
%%% mode: latex
%%% TeX-master: "n"
%%% End:
