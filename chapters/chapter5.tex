Galois fields are the finite fields. They can be completely characterized in terms of splitting fields of some polynomials. It is found that the Galois group of an extension of a finite field by a finite field is cyclic.

\begin{definition}[Prime Fields]
Let \(F\) be a field and let \(P\) be the intersection of all subfields of \(F\). Then \(P\) is a field with no proper subfields. This field \(P\) is called the Prime subfield of \(F\).
\end{definition}

\begin{enumerate}
\item If \(charF=p(prime)\), then \(P\cong {\mathbb{Z}}_p\).
\item If \(charF=0\), then \(P\cong \mathbb{Q}\).
\end{enumerate}

\begin{theorem}
A  finite field \(F\) has \(p^n\) number of elements where \(p \in \mathbb{Z}_+\) is a prime and it has \(p^n\) number of elements if and only if \(F\) is a splitting field of \(x^{p^n} - x\) over \(\mathbb{Z}_p\).\\
\end{theorem}

\begin{theorem}
  If \(F\) is a finite dimensional extension field of a finite field \(K\), then \(F\) is finite and is Galois over \(K\). The Galois group \(Aut_K^F\) is cyclic.
\end{theorem}

\section{Representation of Finite Fields}
Basically there are two types of representation of a finite field. These two representations are equivalent. 
\subsection{Integer representation}

\(GF(p^n)=\{0,1,...,p-1\} \cup \{p,p+1,...,p+p-1\} \cup ... \cup \{p^{n-1},p^{n-1}+1,...,p^{n-1}+p^{n-2}+...+p-1\}\).

\begin{example}
    \(GF(2)=\{0,1\}\)\\
    \(GF(2^3)=\{0,1\} \cup \{2,2+1\} \cup \{2^2,2^2+1,2^2+2,2^2+2+1\}=\{0,1,2,3,4,5,6,7\}\)
\end{example}

But the digits \(2,3,..,7\) of the field \(GF(2^3)\) do not lie on the field \(GF(2)\). If we look the field \(GF(2^3)\) as an extension field of \(GF(2)\) and write its elements using only the elements of the base field \(GF(2)\) then we have the following representations:
\vspace{3mm}

\begin{tabular}{|c|c|c|}
    \hline
    Digits & \ & Binary rep..\\
    \hline
    3 & \(2+1\) & 011 \\
    4 & \(2^2+2^1 \times 0 +2^0 \times 0\) & 100 \\
    5 & \(2^2+1\) & 110 \\
    \hline
\end{tabular}
\vspace{3mm}

This is actually \textbf{Binary representation} of the field \(GF(2^3)\)



\subsection{Polynomial representation}
For a field \(F\) and and an irreducible polynomial \(f(x) \in F[x]\) the quotient ring \(F[x]/(f(x))\) is field.\\
If \(F\) is a finite field consisting of \(p\) number of elements and \(f(x) \in F[x]\) is irreducible then \(F[x]/(f(x))\) is finite field. This field consists of all polynomials modulo \(f(x)\). If \(F=GF(2^3)\) then \(x^8+x^7+...+x+1 \in F[x]\) is irreducible in \(F[x]\). Since \(F\) has \(8\) elements which are modulo \(8\), \(F\) is represented by the factor ring \(F[x]/(f(x))\). \\


In the example above, the number \(5\) has the representation \(2^2+1\). This gives the polynomial representation \(x^2+1=(1,0,1)\)(coefficient of \(x^2\) is \(1\) of \(x\) is \(0\) and of constant is \(1\) ) Now the binary equivalent of \(5\) is \(101\).

\section{Operations in Galois Field}
Let the Galois field be \(GF(p^n)\). Since the elements of a Galois field can be represented as polynomials the operations are similar to polynomial operations. Let \(f(x)=a_0+a_1x+..+a_{n-1}x^{n-1}\) and \(g(x)=(b_0+b_1x+...+b_{n-1}x^{n-1}\).
\begin{enumerate}
  \item Addition \\
  \(f(x)+g(x)\;\;\;\; (modp)\) 
  
  \item Multiplicaiton \\
  \(f(x).g(x)\;\;\; (modp)\)
\end{enumerate}


