\begin{enumerate}
\item Is every polynomial equation solvable by the method of radicals?
\item Equivalently, does there exist an explicit "formula" which gives all solutions of a polynomial equation?
\end{enumerate}

If the degree of  the polynomial \(f\) is at most four then the answer is \textbf{yes} \cite{hunger}.\\
For example, the quadratic polynomial equation \(ax^2+bx+c=0\) has the formula \[x=\frac{-b \pm \sqrt{b^2-4ac}}{2a}\]
\vspace{3mm}

\section{Formulation of the Classic Problem}
The formula by the method of radicals means the ``formula involving only field operations and the extraction of \(nth\) roots'' \cite{hunger}.
``The existence of a formula means there is a finite sequence of steps, each step being a field operation or the extraction of an \(nth\) roots, which yields all solutions of the given polynomial.
Performing a field operation leaves the base field unchanged, but the extraction of an \(nth\) root of an element
\(c\) in a field \(K\) amounts to constructing an extension field \(K(u)\) with \(u^n \in K\). Thus the existence of a formula for solving \(f(x)=0\) would imply
the existence of a finite tower of fields
\[K=E_0 \subset E_1 \subset ... \subset E_n\]
such that \(E_n\) contains a splitting field of \(f\) over \(K\) and for each \(i \geq 1\), \(E_i=E_{i-1}(u_i)\) with some positive power of \(u_i\) lying in \(E_{i-1}\)'' \cite{hunger}.\\
``Conversely suppose there exists such a tower of fields and that \(E_n\) contains a splitting field of \(f\). Then
\[E_n = K(u_1,u_2,...,u_n)\]
and each solution is of the form \(f(u_1,...,u_n)/g(u_1,...,u_n)\) where \(f,g \in K[x_1,...,x_n]\).
\clearpage

Thus each solution is expressible in terms of a finite number of elements of \(K\), a finite number of field operations and \(u_1,...,u_n\). But
this amounts to saying that there is a formula for the solutions of the particular given equation'' \cite{hunger}.
\vspace{3mm}

\begin{definition} \cite{hunger} [Radical Extension]
An extension field \(F\) of a field \(K\) is a radical extension of \(K\) if \(F=K(u_1,...,u_n)\), some power of \(u_1\) lies in \(K\) and for each \(i \geq 2\), some power of \(u_i\) lies in \(K(u_1,...,u_{i-1})\).
\end{definition}

\vspace{3mm}
\begin{remark} \cite{hunger}
If \(u_i^m \in K(u_1,...,u_{i-1})\) then \(u_i\) is a root of \(x^m-u_i^m \in K(u_1,...,u_{i-1})[x]\). \\
Therefore every radical extension \(F\) of \(K\) is finite dimensional algebraic over \(K\).
\end{remark}
\vspace{3mm}

\begin{definition} \cite{hunger}
  The polynomial equation \(f(x)=0\) is \textit{solvable by radicals} if there exists a radical extension \(F\) of \(K\) and splitting field \(E\) of \(f\) over \(K\) such that \(F \supset E \supset K\).
\end{definition}

\begin{theorem} \cite{hunger}
If \(F\) is a radical extension of \(K\) and \(E\) is an intermediate field, then \(Aut_K^E\) is a solvable group.
\end{theorem}

\begin{corollary} \cite{hunger}
If the polynomial equation \(f(x)=0\) is solvable by radicals, then the Galois group of \(f\) is a solvable group.
\end{corollary}
\vspace{3mm}

\section{Group Theoretic Concepts}
Let \(G\) be a group.

\begin{definition} \cite{hunger} [Solvable Series]
A finite chain of subgroups \(G=G_0>G_1>...>G_n={e}\) such that \(G_{i+1}\) is normal in \(G_i\) for \(0 \leq i < n\) is called a subnormal series of \(G\).\\
A subnormal series is a solvable series if each factor group \(G_i/G_{i+1}\) is abelian.
\end{definition}

\begin{definition} \cite{hunger} [Solvable Group]
 A group is solvable if and only if it has a solvable series.
\end{definition}

If \(F\) is a radical extension of \(K\) then \(F\) is Galois over \(K\) and by the Fundamental Theorem of Galois \(Aut_K^E\) has a solvable series where \(E\) is an intermediate field. So \(Aut_K^E\) is solvable.
\clearpage

\section{Outcomes}
\begin{theorem} \cite{hunger}
The symmetric group \(S_n\) is not solvable for \(n \geq 5\).
\end{theorem}

The polynomial \(f(x)=x^5-10x+5 \in \mathbb{Q}[x]\) has Galois group ``\(S_5\), which is not a solvable group'' \cite{hunger}. \\[3mm]
The quantic polynomial equations over \(\mathbb{Q}\) are not solvable by radicals. That is there does not exist an explicit formula for solving the quantics. \\
Moreover, ``polynomial equations of degree \(n \geq 5\) are not solvable by radicals'' \cite{hunger}.

\subsection{Illustrations}
Galois theory gives the precise condition under which a polynomial of degree \(n \geq 5\) is solvable by radicals or not.

\vspace{5mm}
\begin{example}
  The polynomial is \(g(x)=x^5-1 \in \mathbb{Q}[x]\).\\
  The set of  roots of this polynomial equation \(g(x)=0\) are the fifth roots of unity which forms a group under addition modulo \(5\). Hence the ``Galois group is isomorphic to \(\mathbb{Z}_5\)'' \cite{hunger}. The group \(\mathbb{Z}_5\) is cyclic and ``every cyclic group is solvable'' \cite{galois}. Hence this polynomial is solvable by radicals.
\end{example}

\vspace{3mm}
\begin{definition} \cite{galois} [Cyclotomic Polynomial]
  The \(nth\)-cyclotomic polynomial is the polynomial \({\Phi}_n\) defined as \({\Phi}_n= \prod {(x-\zeta)}\), where \(\zeta\) is a primitive-\(nth\) of unity.
\end{definition}

\vspace{3mm}
\begin{theorem} \cite{galois}
  The Galois group of a \(nth\)-cyclotomic polynomial \({\Phi}_n\) of is \(\mathbb{Z}_n\).
\end{theorem}

\vspace{3mm}
\begin{example}
The polynomial is \(f(x)=x^{12}-x^{10}+x^8-x^6-x^2+1 \in \mathbb{Q}\)\\
which is a 58th-cyclotomic polynomal i.e this polynomial \(f(x)={\Phi}_{58}\).\\
  So its Galois group is \(\mathbb{Z}_{58}\), which is abelian and hence is solvable. Therefore this polynomial equation \(f(x)=0\) is solvable by radicals.
\end{example}
