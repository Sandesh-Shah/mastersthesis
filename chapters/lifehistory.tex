The life of \textit{Évariste Galois} is the most famous, fascinating, and commented life of any mathematician \cite{galois}. There is enough gaps in our knowledge of Galois' life to allow imaginations which has made it fascinating.\\

\textit{Évariste Galois} was born on Oct 25, 1811 near Paris. His father was a mayor and the director of a school and his mother as the daughter of magistrate. In 1823(age=12) he entered the royal school of Louis-le-Grand. He was commented to be brilliant but also rebellious and some what bizarre by his teachers. He read the whole of Legendre's \textit{Elements of geometry},  Lagrange's texts on the resolution of equations, and works by Euler, Gauss, and Jacobi.\\
In 1827(age=16) he obtained \textbf{first prize} in a national mathematics competition; and the following year he obtained \textbf{accessit}(best papers after the first, second and third prizes) in the same competition and as well as two accessits  in Greek \cite{galois}.\\

An article he sent to the ``Academy of Sciences'' was lost by ``Cauchy'' in 1829(age=18). It was at this point in his life he started facing challenges in his life. After few months his father committed suicide. A few days later, he had entrance examination to the Ecole Polytechnique which he failed badly and had to be in a school of much lower level. He then submitted a paper to Fourier for the ``Grand Prize in Mathematics'' of the Academy of Sciences. Fourier died shortly and the paper was lost and the Prize was awarded to Abel and Jacobi.\\

He then begun to live a political life as intense as his mathematical one \cite{galois}. In the July of 1830 he and the students were locked into their school to prevent them from participating in the action outside. In January of 1831 he was expelled from the school. On July 4, 1831 Poisson and Lacroix published their report on Galois memoir, commenting the text of his is extremely difficult to understand, and the author could certainly have provided additional explanations.\\

On July 14, Galois was arrested at the head of a large group of demonstrators and was sent to prison for six months. Because of cholera epidemic, in 1832, Galois was transferred to another place. There he has a brief love affair with a young woman, Stephanie D. This appears to have been the cause of the duel he fought few days later which became the cause for his death \cite{galois}.
